\chapter{Формулы для вычисления величин}

$$\vec x=(x_1, ..., x_n)$$

\begin{enumerate}
	\item \textbf{Максимальное значение выборки}
	$$M_{\max} = \max\{x_1, .., x_n\}$$
	\item \textbf{Минимальное значение выборки}
	$$M_{\min} = \min\{x_1, .., x_n\}$$
	\item \textbf{Размах выборки}
	$$R = M_{\max} - M_{\min}$$
	\item \textbf{Выборочное среднее (математическое ожидание)}
	$$\hat \mu(\vec x) = \frac{1}{n} \sum_{i=1}^n x_i$$
	\item \textbf{Состоятельная оценка дисперсии}
	$$S^2 (\vec x) = \frac{1}{n-1} \sum_{i=1}^n (x_i - \overline x)^2,$$
	
	где $ \overline{x} = \hat \mu $
\end{enumerate}

\chapter{Определение эмпирической плотности и гистограммы}

\textbf{Эмпирической плотностью} (отвечающей выборке $\vec x$) называют функцию

\begin{equation*}
	\hat f_n(x) =
	\begin{cases}
		\frac{n_i}{n \Delta}, x \in J_i, i = \overline{1; p} \\
		0, \text{ иначе} \\
	\end{cases},
\end{equation*}

где $(J_i, n_i)$ -- интервальный статистический ряд

\hfill

Пусть $\vec x$ -- выборка из генеральной совокупности $X$. Если объем $n$ этой выборки велик, то значения $x_i$ группируют не только в статистический ряд, но и в интервальный статистический ряд. Для этого отрезок
$J = [x_{(1)}, x_{(n)}]$ (где $x_{(1)}=\min\{x_1,..,x_n\}$, $x_{(n)}=\max\{x_1,..,x_n\}$) делят на $p$ равновеликих частей:

\begin{equation*}
	J_i = [a_i, a_{i+1}), i = \overline{1; p - 1}
\end{equation*}

\begin{equation*}
	J_{p} = [a_{p}, a_{p+1}]
\end{equation*}

$$a_i = x_{(1)} + (i-1)\cdot\Delta, i = \overline{1;p+1}$$

$$\Delta = \frac{|J|}{p} = \frac{x_{(n)} - x_{(1)}}{p}$$

Интервальным статистическим рядом называют таблицу

\begin{table}[h!]
	\centering
	\begin{tabular}{|c|c|c|c|c|}
		\hline
		$J_1$ & ... & $J_i$ & ... & $J_p$ \\
		\hline
		$n_1$ & ... & $n_i$ & ... & $n_p$ \\
		\hline
	\end{tabular}
\end{table}

Здесь $n_i$ -- количество элементов выборки $\vec x$, которые
$\in J_i$

\hfill

В нашем случае $p=m=[\log_2n] +2$

\textbf{Гистограммой} называют график эмпирической плотности. 

\chapter{Определение эмпирической функции распределения}

Пусть $\vec x = (x_1, ..., x_n)$ -- выборка из генеральной совокупности $X$. Обозначим $n(x, \vec x)$ -- чисор элементов вектора $\vec x$, которые имеют значения меньше $x$.

\hfill

\textbf{Эмпирической функцией распределения} называют функцию 

$F_n : \mathbb{R} \to \mathbb{R}$, определенную условием $F_n(x) = \frac{n(x, \vec x)}{n}$. 

\chapter{Текст программы}

\hfill 

\inputminted[
frame=single,
framesep=2mm,
baselinestretch=1.2,
bgcolor=white,
fontsize=\footnotesize,
linenos,
breaklines
]{matlab}{../src/main.m}


\chapter{Результаты расчетов для выборки (вариант 17)} 

\begin{enumerate}
	\item Максимальное значение выборки
	\item Минимальное значение выборки
	\item Размах выборки
	\item Выборочное среднее (математическое ожидание)
	\item Состоятельная оценка дисперсии
	\item Группировка значений выборки в $m = [log_2 n] + 2$ интервала
	
	\img{300pt}{vals}{Группировка значений}
	\clearpage
	\item Гистограмма и график функции плотности распределения вероятностей нормальной случайной величины с математическим ожиданием $\hat \mu$ и дисперсией $S^2$. 
	
	\img{250pt}{hist}{Гистограмма}
	
	\item График эмпирической функции распределения и функции распределения нормальной случайной величины с математическим ожиданием $\hat \mu$ и дисперсией $S^2$. 
	
	\img{250pt}{plot}{Функции распределения}
	
\end{enumerate}